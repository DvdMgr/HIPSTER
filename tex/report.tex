\documentclass[10pt,onecolumn]{article}
%%%%%%%%%%%%%%%%%%%%%%%%%
%			AZ' STANDARD NEWCOMMANDS
%%%%%%%%%%%%%%%%%%%%%%%%%
\usepackage[english]{babel}
\usepackage[T1]{fontenc}
\usepackage{cite, url,color} % Citation numbers being automatically sorted and properly "compressed/ranged".
%\usepackage{pgfplots}
\usepackage{graphics,amsfonts}
\usepackage[pdftex]{graphicx}
\usepackage[cmex10]{amsmath}
% Also, note that the amsmath package sets \interdisplaylinepenalty to 10000
% thus preventing page breaks from occurring within multiline equations. Use:
 \interdisplaylinepenalty=2500
% after loading amsmath to restore such page breaks as IEEEtran.cls normally does.
\usepackage[utf8]{inputenc}
% Useful for dispalying quotations
\usepackage{csquotes}
% Compact lists
\usepackage{enumitem}

\usepackage{array}
% http://www.ctan.org/tex-archive/macros/latex/required/tools/
\usepackage{mdwmath}
\usepackage{mdwtab}
%mdwtab.sty	-- A complete ground-up rewrite of LaTeX's `tabular' and  `array' environments.  Has lots of advantages over
%		   the standard version, and over the version in `array.sty'.
% *** SUBFIGURE PACKAGES ***
\usepackage[tight,footnotesize]{subfigure}

\usepackage[top=1.5cm, bottom=2cm, right=1.6cm,left=1.6cm]{geometry}
\usepackage{indentfirst}

\usepackage{times}
% make sections titles smaller to save space
\usepackage{sectsty}
\sectionfont{\large}
% enable the use of 'compactitem', a smaller 'itemize'
\usepackage{paralist}

\setlength\parindent{0pt}
\linespread{1}

\def\C#1{\mathcal{#1}}

\renewcommand{\phi}{\varphi}
% OPERATORS
\newcommand{\floor}[1]{{\left\lfloor #1\right\rfloor}}
\newcommand{\ceil}[1]{{\left\lceil #1\right\rceil}}
\newcommand{\E}[1]{\mathop{\rm E}\nolimits\left[#1\right]} % Expectation
\newcommand{\pr}[1]{\Pr\left[ #1 \right]}
\renewcommand{\P}[1]{\mathrm{P}\left[ #1 \right]}

\newcommand{\mbb}{\mathbb}
\newcommand{\mee}{\mathrm{e}}
\newcommand{\mrr}{\mathrm}
\newcommand{\m}{\mathrm{m}}
\newcommand{\mmin}[1]{{\min\left\{#1\right\}}}
\newcommand{\mmax}[1]{{\max\left\{#1\right\}}}
\newcommand{\id}[1]{\mathbf{1}\(#1\)} % Unit function
\newcommand{\Heav}[1]{H\left(#1\right)} % Heaviside function
\newcommand{\eps}{\varepsilon}
\newcommand{\rect}[1]{\mathrm{rect}\!\left(#1\right)}
\newcommand{\sinc}[1]{\mathrm{sinc}\!\left(#1\right)}
\newcommand{\bin}[2]{{\left(\begin{array}{c}#1\\#2\end{array}\right)}}

\newcommand{\me}[2]{\left[ {#1} \right]_{(#2)}} % Submatrix \me{P}{i,j} produces [P]_{i,j} and denotes the element in the i-th row and j-th column of P
\newcommand{\vv}[1]{\left[ {#1} \right]} % Submatrix \vv{a,b,c} produces [a,b,c]

\newcommand{\Set}[1]{{\C #1}}
\newcommand{\Setd}[2]{\Set{#1}=\left\{#2\right\}}  % Set definition: \Set{S}{0,1,2} produces S={01,2,} where S is in mathcal

\newcommand{\mat}[1]{{\hbox{\textbf{#1}}}}
\newcommand{\ei}[1]{{\mat{e}_{#1}}}   % all zero vector except in the $#1$-th element which is one
\newcommand{\ind}[1]{\mathbf{\chi}\left\{#1\right\}} % Indicator function \ind{A}=1 if A is true, \ind{A}=0 otherwise.

% FORMATTING
\newcommand{\ie}{i.e.,\,}
\newcommand{\eg}{e.g.,\,}
\newcommand{\columnbreak}{\vfill\eject} % Column break

% REFERENCES
\newcommand{\Fig}[1]{Fig.~\ref{#1}}
\newcommand{\eq}[1]{(\ref{#1})}
\newcommand{\Tab}[1]{Tab.~\ref{#1}}
\newcommand{\Sec}[1]{Sec.~\ref{#1}}

\begin{document}
\title{HIPSTER, the alternative file transfer protocol}
\author{}
\date{}
\maketitle

\section{Protocol description}
\begin{figure}[htp]
	\centering
    \includegraphics[height=0.15\textheight]{tex/images/packet_structure.pdf}
	\caption{HIPSTER packet structure}
	\label{fig:header}
\end{figure}
The packet structure has been designed in order to add the minimum overhead,
while providing all the functionality needed by the protocol.

Destination IP and port are needed for packet forwarding within the channel
and by the receiver for replying to the sender. Moreover the size of the
payload was included to ease the parsing of the packet at the receiving side.
Finally \texttt{CODE} and \texttt{SEQUENCE NUMBER} are used for signalling
and error recovery.

The following values are used for the code:
\begin{compactitem}
\item 0 means a regular data packet
\item 1 carries an ACK, the sequence number received is the same as the packet
	being ACKed
\item 2 signals the end of transmission (ETX). It is issued by the sender and
	ACKed by the receiver
\end{compactitem}

Define DR, DS, TC\\
Compare HIPSTER with stop and wait and with a simple delay between packets.\\
Tell them we tried to use the delay before the poor man's sliding window.
\begin{figure}[htp]
  \centering
  \subfigure[DS flowchart]{\includegraphics[width=0.45\linewidth, keepaspectratio]{Documentation/sender_flowchart.pdf}}
  \subfigure[DR flowchart]{\includegraphics[width=0.45\linewidth, keepaspectratio]{Documentation/receiver_flowchart.pdf}}
  \caption{HIPSTER protocol flowchart}
  \label{fig:flowchart}
\end{figure}

\section{Transport Channel test results}
The Transport Channel (TC) module simulates a bad channel among the DS and DR. It drops received UDP packets of length $L$ with probability $P_{drop} = 1 - \exp(-L/1024)$ and forward the remaining ones with a delay distribuited according to an expontial random variable with mean $1024/\ln(L)$. \\
The TC was tested in loopback, by measuring the delay between the transmission of a packet by DS and its reception at DR and the ratio between sent and received packets. The payload of UDP packet used in this test was in the range from 12 byte (HIPSTER header length, thus the size of an ACK) to 1012 byte (actual size of HIPSTER packets). Each measurement was taken 10 times. The results are in figure~\ref{fig:TCstats}. The TC follows accurately the theoretical model, the few discrepancies are related to the Java random number generator and the finiteness of measurements. The distribution of the delays introduced by the TC in a transmission fits the required exponential RV according to the Kolmogorov-Smirnov test and the comparison of the cumulative distributive functions (CDF) of the two can be seen in figure~\ref{fig:CDF}.
%this second figure can be omitted
\begin{figure}[htp]
  \centering
  \subfigure[Dropping probability]{\includegraphics[width=0.45\linewidth, keepaspectratio]{tex/images/pdrop.pdf}}
  \subfigure[Mean delay]{\includegraphics[width=0.45\linewidth, keepaspectratio]{tex/images/delayChannel.pdf}}
  \caption{TC statistics}
  \label{fig:TCstats}
\end{figure}
\begin{figure}[htp]
  \centering
  \includegraphics[width = 0.45\linewidth, keepaspectratio]{tex/images/cdfChannel.pdf}
  \caption{CDF of the delays introduced by TC, $L_{UDP} = 1012$ byte}
  \label{fig:CDF}
\end{figure}


\section{Performance analysis}


\section{Concusion}
O RLY? YUP!
\end{document}
